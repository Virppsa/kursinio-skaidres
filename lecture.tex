% Copyright 2004 by Till Tantau <tantau@users.sourceforge.net>.
%
% In principle, this file can be redistributed and/or modified under
% the terms of the GNU Public License, version 2.
%
% However, this file is supposed to be a template to be modified
% for your own needs. For this reason, if you use this file as a
% template and not specifically distribute it as part of a another
% package/program, I grant the extra permission to freely copy and
% modify this file as you see fit and even to delete this copyright
% notice. 

\documentclass{beamer}

\usetheme{Madrid}
\usepackage[english]{babel}
\usepackage[utf8x]{inputenc}
\def\LTfontencoding{L7x}
\PrerenderUnicode{ąčęėįšųūž}
\usepackage{times}
\usepackage[T1]{fontenc}
\usepackage{color}
\usepackage{verbatim}
\usepackage{graphicx}
\usepackage{fancyvrb}
\usepackage{bm}
\usepackage{amsfonts}
\usepackage{float}
\usepackage{hyperref}

\include{pythonlisting}


\title{Mano pristatymo pavadinimas}
\author{Vardenis Pavardenis}
\institute[Vilniaus Universitetas]
{
  \inst{1}%
  Matematikos ir informatikos institutas\\
  Vilniaus universitetas
}
\date{2014}



% Let's get started
\begin{document}

\begin{frame}
  \titlepage
\end{frame}

\begin{frame}{Turinys}
  \tableofcontents
\end{frame}

\section{LaTeX konstrukcijų pavyzdžiai}
\subsection{Sąrašai}
\begin{frame}{Sąrašai}
\begin{itemize}
    \item Nenumeruojamo sąrašo pavyzdys
    \item Antras punktas
\end{itemize}

\begin{enumerate}
    \item Numeruojamo sąrašo pavyzdys
    \item Sekančioje skaidrėje pateikiamas vu logotipas (žr. \ref{fig:vu logo}).
\end{enumerate}
\end{frame}

\subsection{Paveiksliukai}
\begin{frame}{Paveiksliuko įterpimo pavyzdys}
\begin{figure}[!htbp]
    \includegraphics[scale=0.43]{vu_logo}
    \label{fig:vu logo}

    \caption{VU logotipas}
\end{figure}
\end{frame}

\subsection{Python kodo fragmento pavyzdys}
\begin{frame}[fragile]{Python kodo framgmento įterpimo pavyzdys}
% Python kodo fragmentus galima įterpti štai taip:
Šis fragmentas yra ne 2 kart 2 o, 2 pakelta kvadratu:
\begin{python}
>>> 2**2
4
\end{python}
\end{frame}


\appendix
\section<presentation>*{\appendixname}
\subsection<presentation>*{For Further Reading}

\begin{frame}[allowframebreaks]
  \frametitle<presentation>{Šaltiniai}
  \begin{thebibliography}{10}
    
  \beamertemplatebookbibitems

  \bibitem{Author1990}
    Mano ypatingas svarbus šaltinis
    \newblock {\url{http://python.org/dev/peps/pep-0008}}.
 
  \end{thebibliography}
\end{frame}

\end{document}
